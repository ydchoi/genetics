\documentclass{article} % For LaTeX2e
\usepackage{nips14submit_e,times}
\usepackage{amsmath}
\usepackage{amsthm}
\usepackage{amssymb}
\usepackage{mathtools}
\usepackage{hyperref}
\usepackage{url}
\usepackage{algorithm}
\usepackage[noend]{algpseudocode}
%\documentstyle[nips14submit_09,times,art10]{article} % For LaTeX 2.09

\usepackage{graphicx}
\usepackage{caption}
\usepackage{subcaption}

\def\eQb#1\eQe{\begin{eqnarray*}#1\end{eqnarray*}}
\def\eQnb#1\eQne{\begin{eqnarray}#1\end{eqnarray}}
\providecommand{\e}[1]{\ensuremath{\times 10^{#1}}}
\providecommand{\pb}[0]{\pagebreak}

\newcommand{\E}{\mathrm{E}}
\newcommand{\Var}{\mathrm{Var}}
\newcommand{\Cov}{\mathrm{Cov}}

\def\Qb#1\Qe{\begin{question}#1\end{question}}
\def\Sb#1\Se{\begin{solution}#1\end{solution}}

\newenvironment{claim}[1]{\par\noindent\underline{Claim:}\space#1}{}
\newtheoremstyle{quest}{\topsep}{\topsep}{}{}{\bfseries}{}{ }{\thmname{#1}\thmnote{ #3}.}
\theoremstyle{quest}
\newtheorem*{definition}{Definition}
\newtheorem*{theorem}{Theorem}
\newtheorem*{lemma}{Lemma}
\newtheorem*{question}{Question}
\newtheorem*{preposition}{Preposition}
\newtheorem*{exercise}{Exercise}
\newtheorem*{challengeproblem}{Challenge Problem}
\newtheorem*{solution}{Solution}
\newtheorem*{remark}{Remark}
\usepackage{verbatimbox}
\usepackage{listings}
\title{Human Genetics: \\
Problem Set I}


\author{
Youngduck Choi \\
CILVR Lab \\
New York University\\
\texttt{yc1104@nyu.edu} \\
}


% The \author macro works with any number of authors. There are two commands
% used to separate the names and addresses of multiple authors: \And and \AND.
%
% Using \And between authors leaves it to \LaTeX{} to determine where to break
% the lines. Using \AND forces a linebreak at that point. So, if \LaTeX{}
% puts 3 of 4 authors names on the first line, and the last on the second
% line, try using \AND instead of \And before the third author name.

\newcommand{\fix}{\marginpar{FIX}}
\newcommand{\new}{\marginpar{NEW}}

\nipsfinalcopy % Uncomment for camera-ready version

\begin{document}


\maketitle

\begin{abstract}
This work contains the solutions to the problem set I
of Human Genetics 2015 course at New York University.
\end{abstract}

\begin{question}[1] 
\end{question}

\smallskip

\begin{solution}
\textbf{a.} As the gamete from the $YY$ pea must be $Y$, the peas in the $F_1$ generation
must contain at least $1$ $Y$ allele. Since we are given that $YY$ and $Yy$ genotypes result
in yellow color, we have that the peas in the $F_1$ generation must be yellow. In other words,
the expected frequency of yellow peas in the $F_1$ generation of a cross between $YY$ and $yy$
is $1$.

\smallskip

\textbf{b.} Notice that the argument of 
The expected frequency of yellow peas in 

\smallskip

\textbf{c.}

\smallskip

\end{solution}

\bigskip


\begin{question}[2]
\end{question}
\begin{solution}
\end{solution}

\bigskip

\begin{question}[3]
\end{question}
\begin{solution}
\textbf{a.} As the father is type $AB$, we know that his genotype is $I^A I^B$. For the case of the mother,
since $O$ is the recessive trait, her genotype is $I^O I^O$. \\

\smallskip

\textbf{b.} The genotype of their children can be either $I^A I^O$ or $I^B I^O$. Since 
$I^A$ and $I^B$ are both dominant to $I^O$, the phenotype of their children can be
either $A$ or $B$.

\smallskip

\textbf{c.} Since the father is type $A$, he can have either $I^A I^A$ or $I^A I^O$ 
for his genotype. As the mother is type $B$, she can have either $I^B I^B$ or $I^B I^O$.
Hence, the possible blood types among their children are $AB$, $A$, $B$, and $O$.


\end{solution}

\bigskip

\begin{question}[4]
\end{question}
\begin{solution}
\textbf{a.} 
Notice that none of the parents possess $I^B$ allele. As the $AB$ phenotype requires 
a possession of $I^B$ allele, their first child cannot have the phenotype $AB$ for the
$I$ locus. Hence, the probability that their first child will have the phenotype $AB+$ is
$0$.

\smallskip

\textbf{b.} Notice that one parent has $DD$ genotype for the $Rh$ locus. Hence, their first
child will always possess a $D$ allele, which makes the recessive trait $Rh$negative not a
possibility. Therefore, the probability that their first child will have the phenotype $A-$
is $0$.

\smallskip

\textbf{c.} Hence, the probability that their first child will have the phenotype $A+$ is $1$.


\end{solution}


\end{document}

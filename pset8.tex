\documentclass{article} % For LaTeX2e
\usepackage{nips14submit_e,times}
\usepackage{amsmath}
\usepackage{amsthm}
\usepackage{amssymb}
\usepackage{mathtools}
\usepackage{hyperref}
\usepackage{url}
\usepackage{algorithm}
\usepackage[noend]{algpseudocode}
%\documentstyle[nips14submit_09,times,art10]{article} % For LaTeX 2.09

\usepackage{graphicx}
\usepackage{caption}
\usepackage{subcaption}

\def\eQb#1\eQe{\begin{eqnarray*}#1\end{eqnarray*}}
\def\eQnb#1\eQne{\begin{eqnarray}#1\end{eqnarray}}
\providecommand{\e}[1]{\ensuremath{\times 10^{#1}}}
\providecommand{\pb}[0]{\pagebreak}

\newcommand{\E}{\mathrm{E}}
\newcommand{\Var}{\mathrm{Var}}
\newcommand{\Cov}{\mathrm{Cov}}

\def\Qb#1\Qe{\begin{question}#1\end{question}}
\def\Sb#1\Se{\begin{solution}#1\end{solution}}

\newenvironment{claim}[1]{\par\noindent\underline{Claim:}\space#1}{}
\newtheoremstyle{quest}{\topsep}{\topsep}{}{}{\bfseries}{}{ }{\thmname{#1}\thmnote{ #3}.}
\theoremstyle{quest}
\newtheorem*{definition}{Definition}
\newtheorem*{theorem}{Theorem}
\newtheorem*{lemma}{Lemma}
\newtheorem*{question}{Question}
\newtheorem*{preposition}{Preposition}
\newtheorem*{exercise}{Exercise}
\newtheorem*{challengeproblem}{Challenge Problem}
\newtheorem*{solution}{Solution}
\newtheorem*{remark}{Remark}
\usepackage{verbatimbox}
\usepackage{listings}
\title{Human Genetics: \\
Problem Set VIII}


\author{
Youngduck Choi \\
New York University\\
\texttt{yc1104@nyu.edu} \\
}


% The \author macro works with any number of authors. There are two commands
% used to separate the names and addresses of multiple authors: \And and \AND.
%
% Using \And between authors leaves it to \LaTeX{} to determine where to break
% the lines. Using \AND forces a linebreak at that point. So, if \LaTeX{}
% puts 3 of 4 authors names on the first line, and the last on the second
% line, try using \AND instead of \And before the third author name.

\newcommand{\fix}{\marginpar{FIX}}
\newcommand{\new}{\marginpar{NEW}}

\nipsfinalcopy % Uncomment for camera-ready version

\begin{document}


\maketitle

\begin{abstract}
This work contains the solutions to the problem set VIII
of Human Genetics 2015 course at New York University.
\end{abstract}

\bigskip

\begin{question}[1]
\end{question}
\begin{solution}
\end{solution}

\bigskip

\begin{question}[2]
\end{question}
\begin{solution}
\textbf{(a)} 5'-CACUAA... $\to$ 5'-CACUAG... would produce a nonsense
mutation, as the UAA base changed to UAG, making amino acid change to STOP.

\smallskip

\textbf{(b)} 5'-CAC... $\to$ 5'-CAU... would produce a silent mutation,
as the CAC base and CAU base have the same amino acid, His. 

\smallskip

\textbf{(c)} 5'-CAC... $\to$ 5'-CCC would produce a missense mutation,
as the amino acid changes from His to Pro.
\hfill $\qed$

\end{solution}

\bigskip

\begin{question}[3]
\end{question}
\begin{solution}
\textbf{(a)} At Hardy-Weinberg equilibrium, the expected frequency 
of color-blindness among women is q^2, where $q$ denotes the frequency
of mutant allele in the population, which in this case is $0.1^2 = 0.01$.

\smallskip

\textbf{(b)} The expected frequency for men is simply $q = 0.1$ as, 
an color-blind man will be hemizygous, requiring exactly one mutant allele.

\hfill $\qed$

 
\end{solution}


\bigskip

\end{document}


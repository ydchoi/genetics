\documentclass{article} % For LaTeX2e
\usepackage{nips14submit_e,times}
\usepackage{amsmath}
\usepackage{amsthm}
\usepackage{amssymb}
\usepackage{mathtools}
\usepackage{hyperref}
\usepackage{url}
\usepackage{algorithm}
\usepackage[noend]{algpseudocode}
%\documentstyle[nips14submit_09,times,art10]{article} % For LaTeX 2.09

\usepackage{graphicx}
\usepackage{caption}
\usepackage{subcaption}

\def\eQb#1\eQe{\begin{eqnarray*}#1\end{eqnarray*}}
\def\eQnb#1\eQne{\begin{eqnarray}#1\end{eqnarray}}
\providecommand{\e}[1]{\ensuremath{\times 10^{#1}}}
\providecommand{\pb}[0]{\pagebreak}

\newcommand{\E}{\mathrm{E}}
\newcommand{\Var}{\mathrm{Var}}
\newcommand{\Cov}{\mathrm{Cov}}

\def\Qb#1\Qe{\begin{question}#1\end{question}}
\def\Sb#1\Se{\begin{solution}#1\end{solution}}

\newenvironment{claim}[1]{\par\noindent\underline{Claim:}\space#1}{}
\newtheoremstyle{quest}{\topsep}{\topsep}{}{}{\bfseries}{}{ }{\thmname{#1}\thmnote{ #3}.}
\theoremstyle{quest}
\newtheorem*{definition}{Definition}
\newtheorem*{theorem}{Theorem}
\newtheorem*{lemma}{Lemma}
\newtheorem*{question}{Question}
\newtheorem*{preposition}{Preposition}
\newtheorem*{exercise}{Exercise}
\newtheorem*{challengeproblem}{Challenge Problem}
\newtheorem*{solution}{Solution}
\newtheorem*{remark}{Remark}
\usepackage{verbatimbox}
\usepackage{listings}
\title{Human Genetics: \\
Problem Set VII}


\author{
Youngduck Choi \\
New York University\\
\texttt{yc1104@nyu.edu} \\
}


% The \author macro works with any number of authors. There are two commands
% used to separate the names and addresses of multiple authors: \And and \AND.
%
% Using \And between authors leaves it to \LaTeX{} to determine where to break
% the lines. Using \AND forces a linebreak at that point. So, if \LaTeX{}
% puts 3 of 4 authors names on the first line, and the last on the second
% line, try using \AND instead of \And before the third author name.

\newcommand{\fix}{\marginpar{FIX}}
\newcommand{\new}{\marginpar{NEW}}

\nipsfinalcopy % Uncomment for camera-ready version

\begin{document}


\maketitle

\begin{abstract}
This work contains the solutions to the problem set VII
of Human Genetics 2015 course at New York University.
\end{abstract}

\bigskip

\begin{question}[1]
\end{question}
\begin{solution}
The second step of the identification requires seeing where
the sequence actually matches in the genome. Determining the mRNA
sequence, therefore, is not adequate.
\end{solution}

\bigskip

\begin{question}[2]
\end{question}
\begin{solution}
\textbf{(a)}
Transcription, RNA Polymerase.

\smallskip

\textbf{(b)}
Splicing, Enhancer.

\smallskip

\textbf{(c)}
Translation, Ribosome, tRNA.

\end{solution}

\bigskip

\begin{question}[3]
\end{question}
\begin{solution}
\textbf{(a)}
NvaII yielded 3 separate pieces of plasmids. This implies that 
the cut happened 2 times. 
\smallskip

\textbf{(b)}
Smaller fragments migrate more rapidly, and Sbfl yields the fragment
that migrated the most rapidly. Hence, Sbfl enzyme yields the smallest 
plasmid fragment.

\end{solution}


\bigskip

\end{document}

